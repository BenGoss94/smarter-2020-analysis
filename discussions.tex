\section{Discussions and policy interpretation}
One of the obstacles in implementing the technologies provided by ICT is lack of knowledge about them. This is especially visible in the developing countries, where communication and the level of government support is lesser than that in developed countries. This leads to much lower penetration of the technologies through the industry. An example of this are agriculture-related solutions, which provide high benefits (such as increased crop yield) to farmers, but aren’t very well known even among the developed countries, let alone the developing ones.

A possible solution to this would be government intervention in terms of direct and indirect education provided for the farmers. Direct education could consist of distributable materials and training, while indirect would comprise the inclusion of information on new ICT solutions into traditional farming equipment. The cost of such actions would be much lower to the policy makers than for example subsidising the purchases of new technologies and will provide incentive to the farmers to be more efficient.


With telecommuting, a tax rebate for using a smaller working space could be beneficial to companies who can't afford to create an ICT network capable of sustaining prolonged telecommuting activity. 

The report states that including information about emissions-reducing technologies on other farm equipment purchases is a viable option, but an even more viable option is having a class or training session that's funded by the government for all farmers to attend that introduces this technology. 

A policy that is recommended by the authors is for a CO$_2$ tax to be implemented, which would then encourage building designers to implement more emission-reducing technology in their designs, and for home owners to add the technology themselves. This policy would work, but what would need to be decided is a rate that would be more than the technology would cost, in order for it to be beneficial for the public to get the technology.

The report states that many countries need to modify the insurance and working regulatory environment, for transport-related sublevers including Electric vehicles and Applications for intermodal travel and public transportation. It also states that the government needs to show the potential of better transport methods more, as the culture needs to change. This call for policy change is accurate, as many countries have a poor uptake in public transport, because getting in their own car is easier. The government should work to change people's opinions more, by running more campaigns in order to change people's opinions on electric vehicles and public transport, and promote them more widely. 

It is suggested that there is no need for policy to drive changes relating to E-paper and E-commerce since these changes will be market driven. It also states the barrier to further adoption is mainly due to people that are reluctant to change or are slow to adapt. This would make it unnecessary to introduce policy to try and force this as it will likely change naturally with time.

The report suggests it is difficult to introduce a policy to force drivers to drive in an eco-friendly way. This is a reasonable assumption as there is no clear way to do this, instead the report proposes better education when learning to drive as well as providing more information to experienced drivers in an attempt to convert their driving style. This would appear to be the most practical way to introduce change.  There is an option to force manufactures to integrate technology into their vehicles however this is likely to have strong opposition as it will increase the price of the vehicle.

To give ranchers economic incentive to invest in livestock management systems, it is suggested policy makers could introduce policy to fairly price land or introduce emissions penalties. Caution must be exercised when introducing penalties especially in developing countries. Policy makers must ensure these penalties don’t rest with the wrong people in the chain and make it impossible to earn a living wage.

The abatement potential for smart water is considered to be so low it is unlikely to be the focus of policy makers. While the abatement is low it could be argued that all reductions are good so it is worth making the effort to introduce policies to incite change. 

Much like E-paper and E-commerce it is suggested that a reduction in inventory will be market driven so it is not necessary to introduce policy. 

