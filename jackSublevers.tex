
\subsection{E-paper}

The E-paper sublever considers the reduction of paper use and printing as a result of E-readers and other technology. This includes paper use, the transport of paper and paper media as well as the emissions associated with printing. 

 The addressable emissions is referenced to the SMART 2020 report which assumes 270 Mt of paper is used globally in 2020, 25\% of which can be eliminated. Neither of these assumptions have any references which gives little confidence in the data. Furthermore, the SMARTer 2020 report uses addressable emissions of 96.6Mt of paper which is not 25\% of 270Mt. The savings potential is referenced to a report, when the report was viewed the percentage given in SMARTer 2020 was not found. This report does consider the impact from a greater use of e-reader devices, so this is incorporated in the SMARTer 2020 calculations.
 
 The main obstacle for E-paper is identified as behavioural barriers, many people will not want to trade physical media for electronic media especially in the older generations. There may also be a reluctance to embrace new technologies among these generations. This is a reasonable assessment of the barriers affecting the uptake of the savings considered in the E-paper sublever. This can be overcome in business by introducing new policies however it could prove challenging to change private use.
 
 \subsection{E-commerce}

The E-commerce sublever is concerned with the buying and selling of products and services over the internet. This sublever targets retail and commerce related transport and commercial building and inventory space.  

 The number used for the addressable emissions has a reference to the IEA without any further indication of a specific report or webpage. The IEA is very large, containing many reports and articles, this made it very difficult to locate the exact origin of the number used. Attempts were made to find the source by searching the website however these attempts were not successful. As previously stated the IEA is a reputable institution and is likely to be a good source for this information. 
 
An assumption is made that 20\% of all private transport is for shopping. This is referenced to the SMART 2020 report which does not provide any further reference for this assumption. A penetration rate of 15\% in the developed world and 0\% in the developing world is assumed, this is referenced to a blog on the wall street journal. This blog was found, however it does not explicitly state a penetration rate of 15\%. The blog gives predictions for internet sales in 2015, which have been extended to 2020 for this report.

 The data suggesting 20\% of all private transport is used for shopping has no support introducing a large uncertainty in any calculations in which it is used. The addressable emissions may have support but the referencing is poor so this cannot be confirmed. The penetration prediction has some support, however the accuracy of the data can be questioned as the method of extrapolation is not given. The assumption of 0\% penetration in the developing world has no support introducing more uncertainty in the calculations. 
 
 As for E-paper the main obstacle for E-commerce is identified as behavioural barriers, in the same way people might not want to swap physical media for electronic media they might also be reluctant to swap shops for online shopping. This provides a reasonable assessment of the barriers to E-commerce.


\subsection{Eco-driving}
Eco-Driving considers drivers adopting a driving style as a result of alerts and other technology to improve the efficiency of their vehicle. 

 This sublever has one reference to SMART 2020 for the abatement potential given. In the SMART 2020 report this sublever was considered under reductions in the logistics sector, for the SMARTer 2020 report this has been moved to the transportation sector. The sublever suggests a 12\% carbon reduction owing to an improved driving style although this number has no support. This leads to an abatement potential of 0.25 GtCO$_2$e. No explanation of this value is offered, the addressable emissions are not stated neither is the savings potential. The lack of support for the numbers used gives the abatement potential very little credibility. The SMART 2020 report provides a list of ‘experts’ consulted for each section, there is no information to suggest which piece of information was provided by which expert. There is also no indication of the information provided by the ‘experts’, while this does add a small amount of credibility it would not be practical to trace all of the experts in an attempt to validate the data. These experts are presented under the Smart Logistics Europe which will provide uncertainty when expanding any predictions to a global level. 
 
There is no clear model presented for the Eco-Driving sublever, a reduction percentage is given and then the answer is presented with no steps between. This makes it impossible to reproduce and further reduces the credibility of the final abatement potential found. 

The main obstacle for Eco-driving is identified as behavioural barriers. It is difficult to convince people to change habits which will limit the uptake. This may be offset by the low cost of implementation and relative ease to put into practise. 


\subsection{Asset sharing/Crowd sourcing}

This sublever is concerned with the knowledge of sharing or reuse of assets through social networks or other communication tools. This targets avoided emissions from reduced manufacturing and extended use. 

 Again the addressable emissions are referenced with IAE, a search of this website was carried out but the numbers in the SMARTer 2020 report could not be found. The savings potential is referenced to a report that was produced by the RAND Corporation. The RAND Corporation is a non-profit global policy think tank, RAND reports undergo rigorous peer review suggesting this is a reliable source of information. The exact value used in the SMARTer 2020 report could not be found in this report, however values were found for similar situations. The way these values were extended to the case covered in SMARTer 2020 was not made clear. This introduces a potential source of error and makes it impossible to reproduce the calculations from the SMARTer2020 report.
 
The RAND report referenced covers a case study in the USA, it was assumed this applied to a global average. This is likely to be an unrealistic assumption due to the relative development of the USA compared with most other countries. The difficulty in finding one reference along with the assumptions made for other values means the accuracy of the result of this sublever can be questioned. 

Asset sharing and crowd sourcing has the obstacle of the high initial cost associated with obtaining a fleet of vehicles. There is also a possible barrier from attitudes towards vehicles not accounted for in the SMARTer 2020 report. People often take pride in their vehicle which may deter them from wanting to share. 

\subsection{Intelligent traffic managment}

This sublever focuses on the use of ICT to remotely monitor and control automobile traffic.

 The addressable emissions are referenced with IEA, the exact numbers could not be found after searching the IEA website. The saving potential has two parts, firstly the fuel wasted due to congestion, and secondly, a penetration rate dependent on the availability of communication technology.
 
 The fuel wasted is referenced to a government website where the number can easily be found. This data is considered to be reliable as it is from a government source. The second part is referenced to a report by the Boston Consulting Group, however the number used in the SMARTer 2020 calculations is not clearly stated. This makes it is impossible to find any evidence for it in the referenced report. Therefore it is very difficult if not impossible to reproduce the calculations in the SMARTer2020 report.
 
 The method used to move from the value for the fuel wasted to the savings potential is not stated. This makes it very difficult to evaluate the accuracy of the abatement potential. The fuel wasted in traffic is an estimate for the USA and is taken to be representative of world congestion. This assumption should be questioned as the well-developed nature of the USA often makes it a bad representation of the global average. The combination of all these uncertainties gives a large margin for error in the final estimate for the abatement potential.
 
 Intelligent traffic management requires a sophisticated communication network, this will be the greatest obstacle preventing the uptake of this system. 


