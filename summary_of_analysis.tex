\section{Summary of analysis}

To evaluate the uncertainty in the abatement estimates four areas of each sublever were analysed. The quality of the data used was considered, this includes the source used as well as the ease of verification based on the quality of the reference. The method used to find the abatement potential from this data was analysed. The obstacles and enabling factors that affect the uptake considered by the SMARTer 2020 report were reviewed as were the system boundaries and assumptions made when calculating the abatement potential.

The data used often lacked detailed references. The references provided didn’t always contain the values used and no explanation was offered of the methods used to find these values. Some of the references provided are too vague, for example in the livestock management sublever the entire EDGAR database is referenced. This offers little help when the size of the database is considered. References to the IEA also lack detail, the reference simply states the IEA, without further detail it is very difficult to find the values due to the high number of publications produced by the IEA. Additionally many of the publications produced by the IEA must be purchased, this further deters any attempt to locate the exact source as a publication must be purchased with no guarantee that is the source used.

 In some cases the model used to generate the abatement potential was not explained, a list of assumptions and a final answer is given with no explanation of the steps taken to find this answer. The Eco-driving sublever is an example of this. This introduces uncertainty as it leaves no way of validating the abatement potential calculated. 
 
The report clearly states the factors limiting the uptake of the sublevers as well as the policies that can be introduced to try and force action. These are generally well considered and cover most of the obvious factors. However, assumptions about the system boundaries can be called into question, the report will take a value for a specific country or countries and extend this to act as a global average with no evidence to suggest this is a valid assumption. 

These factors would suggest that the greatest source of uncertainty is introduced by the quality of the data. The steps taken to extend data from a small scale to a global scale are not always presented. When this is combined with the lack of rigour in the referencing it leaves the value for the abatement potential with enough uncertainty to be questioned. 
