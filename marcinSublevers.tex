\subsection{Livestock managment} 

Livestock management in the agriculture end-use sector, is stated to have the highest abatement potential amongst all the sublevers in agriculture. It targets livestock related emissions, and focuses on reducing those through the process of data collection and communication.

Presumably, the intention of the authors when writing the details for this sublever was to provide a source describing what is meant by, and included in, the livestock-related emissions. The source referenced as “EDGAR database” is broad and without narrowing the source down, it is too vague and doesn’t provide much information. The database contains thousands of entries, and searches for the suitable article, including keywords such as ‘livestock’, ‘emissions’, ‘greenhouse gases’, ‘methane’ and others, have proved fruitless. This does not allow the reproduction of the data given in this SMARTer 2020 sublever.

To compare whether the number given for addressable emissions, 9.93Gt, is of any value, an alternative source has been found. Food and Agriculture Organisation of United Nations (FAO) gives a different value for livestock emissions (7.1 Gt instead of 9.93Gt), as well as precise details on what is included in those emissions. They also provide a different abatement potential from the use of technology (18-30\% rather than 7\%) (Source: FAO, “Tackling Climate Change through Livestock”, Rome 2013). While the numbers differ, the alternative source at least shows that the values are of the same order of magnitude. The difference seems plausible, since the FAO might include different emission factors to the EDGAR database. 

The second source stated in the references for this sublever, “Wireless and the Environment” by BSR, has been searched thoroughly and does not contain information on the split between developed and developing countries. It seems as if the source is mis-referenced, or perhaps the authors had used some sources along with the BSR’s report, and did not include it.

The authors of the report only mention one technology, cattle grazing, which allows for emissions mitigation. The data from FAO’s report suggests that grazing is just one of the few ways in which data collection and communication allows the reduction in air pollution. Feed analyses, forage management and animal management are among those not included in SMARTer 2020. Hence the gap between the 7\% abatement potential stated in GeSI’s report and 18-30\% claimed in FAO’s analysis. 

Using those new estimates from the UN, the abatement potential is between 1.28 and 2.13 GtCO$_2$e for the livestock management sublever (see tool), which is much higher than the 0.70 stated in SMARTer 2020 report.



%\pagebreak


\subsection{Smart water}

Smart water in the agriculture sector has the smallest overall abatement potential in agriculture, but has the highest savings potential value of 25\%. ICT technologies aim to reduce water-related emissions by collecting data about, and making decisions concerning demand and supply trends, to make production and distribution of water more efficient.

First of all, it is not clear what data exactly has been used to estimate the global emissions. The source referenced is “Carbon Footprint of Water” report by River Network and IEA. Table 4.1 in that report suggests that sectors potentially applicable to agriculture in terms of emissions are irrigation and livestock. While livestock has relatively small impact, irrigation comprises of some activities which are agriculture related, like crop growth assistance, and some which aren’t, such as landscape management. There is no further information on how much of irrigation is actually agriculture related and hence there is a large room for error, as irrigation is stated to make up for 20 times as much carbon emissions as livestock.

Secondly, there is the question of what criteria was used to extrapolate US data onto global scale: whether it was per area, per population or by other means. This is unclear and again leaves room for error. An educated guess would be that during such extrapolation, the global carbon emissions would be undervalued, due to US being more developed and hence more energy efficient than the majority of developing countries.

Again, the first source states only carbon dioxide emissions, as opposed to all GHG emissions. This is inaccurate when estimating CO$_2$e emissions.

The last source stated as expert reviews, carries no information on what the reasoning was for choosing 50\% penetration rate. Again this has a large room for error. Additionally if this is an US estimate, then it is likely other less developed and financially capable countries will have much lower penetration rate than the US.


%\pagebreak
\subsection{Reduction in inventory}

Reduction in inventory has got the second biggest abatement potential in the customer and service end-use sector. It mainly aims to reduce the energy used for storing and transporting of goods, by optimising the deliveries.

The reference given in the SMARTer 2020 report redirects a reader to the old version of the report: the SMART 2020. 

There, the considered sublever is classed as a candidate for emissions reduction in the logistics sector, while in the updated report, it is considered a consumer \& services sector.

SMART 2020 provides an appendix with exactly the same text as stated in the sublever details in new report: “24\% reduction in inventory levels; 100\% of warehouses and 25\% of retail are assumed to be used for storage”. 

There are no further references to be traced. A careful reader might notice that the appendix contains a section where experts, who have been consulted for each emissions sector, are listed. While this provides some proof that the assumptions stated in this sublever (and the entire report) have some backing to them, it gives no information as to which industry expert helped with making which assumption. Tracing all 14 experts listed under the logistics section would prove time consuming and difficult. 
Another point to note is that the experts are listed for the logistics section which is titled ‘SMART logistics, Europe’. While the listed specialists are no doubt experts in their field in Europe, the realities on other continents differ significantly. Hence, the figures approximated by the experts probably only apply to European emissions and using the same numbers for estimating global abatement potential gives an even bigger error margin.

It is difficult to judge whether the potential decrease in emissions through reduction in inventory is a significant abatement or not, without knowing the total emissions value for this sublever. It is unclear, however, what reduction in inventory levels mean: whether it is the amount of goods stored per warehouse, or is it reduction of warehouse numbers, or a mix of both. References (1) and (2) may lead to confusion in determining the addressable emissions value. If the simple case is assumed (i.e. that the stated number for abatement potential, 0.18 GtCO$_2$e is 24\% of the addressable emissions) then according to the tool used, it suggests the addressable emissions of 0.75 GtCO$_2$e. The question remains, if and how to incorporate source (2) into this calculation.

There hasn’t been much work done on this sublever in the updated version of the report. The outputs from the SMART2020 have been simply copied over and used without questioning their validity. 



%\pagebreak

\subsection{Virtual power plant}

The virtual power plants have a relatively low abatement potential of just 0.04 GtCO$_2$e, but a high savings potential of 26\% in the power end-use sector. The technology limits the unnecessary electricity generation by system integration of multiple energy generation installations.

The values provided for this sublever state that the savings potential of Virtual Power Plants (VPP’s) is 26\%. Looking at the details, it seems that this has been found by adding the 25\% coming from incorporation of renewables and 1\% from reduction in energy demand. Simply summing up those two values will provide a vague number for estimation. It won’t take into account the fact that incorporation of renewables might already cause reduction in energy demand. This is possible, if for example in a certain region the renewables are more expensive. Then by the economics law of demand, the demand for energy will drop. Hence there is a chance for overlap in those two values, leading to actual reductions being smaller than the given 26\%.

The source has been quoted as ‘Expert interviews’, but the actual expert has not been identified. In the appendix of SMARTer 2020 there is a list of specialists consulted for the sublevers, but only the company they work for is listed. In the SMART 2020, their expertise sector is noted, which for a very curious reader makes it easier to track down which expert has been interviewed for which end-sector. The approach in the SMARTer2020 report is unprofessional and introduces doubt as to whether the specialists were indeed consulted for every sublever that has been presented in the report.

Unfortunately, the access to source (1) is highly priced and only the brochure for that report was available free of charge. %(http://www.navigantresearch.com/wp-content/uploads/2011/11/VPP-11-Brochure.pdf) 

There, one can see that a predicted average global capacity of VPP will reach about 90,000 MW = 90 GW by 2017. This is 3x the value provided in the SMARTer 2020 analysis. 

Yet another alternative source stated that the global VPP market will increase to about 41 GW by 2015. This source quotes the Pike Research’s document when stating this number %(http://www.smartgridnews.com/artman/uploads/1/Virtual_Power_Plant_Capacity_to_Double_in_Size_by_2015.pdf)

While the three documents disagree, there seems to be evidence that the future capacity by 2020 will be higher than the one stated in SMARTer 2020 report. 

Assuming the emissions per 1GW would be the same whether total emissions were 30 or 90GW, then using the tool, is has been calculated that based on the SMARTer 2020 sources, emissions per 1 GW are 0.00467 GtCO$_2$e. Again, without access to the original article, it will be impossible to determine how this number has been estimated.


%\pagebreak

\subsection{Optimization of truck route planning}

Optimization of truck route planning in the transportation end-use sector aims at reducing the emissions coming from commercial road transportation by improving the efficiency of delivery trucks. 

The value provided for the addressable emissions, 3.88 GtCO$_2$e, appears also in other sublevers concerning the road transportation. The source on which it is based is International Energy Agency. This is a highly reputable institution and likely to be one of the best sources for this kind of emissions data. 

The exact source is not referenced. This could be a report by IEA, a table or chart containing the data in concise form, or any other way in which IEA presents its data. This lack of precision in referencing makes it harder for the reader to identify the relevant document/statistics and to replicate the information presented in the SMARTer 2020 report.

An additional difficulty is that the data published by IEA needs to be purchased before accessing it. Hence, even if a curious reader found a few articles which might have contained the required data, they would need to pay to read each one before finding the data, without any guarantee for success. This practice by SMARTer 2020 authors highly discourages any attempts to replicate and validate the data presented.

The savings potential value provided, 0.19 GtCO$_2$e is referenced to the SMART 2020 report. In that report, however, the number stated is 0.100 GtCO$_2$e, so it’s almost half of the updated one. No extra information has been provided in SMARTer 2020 as to why the number used is different. Perhaps the future impact and penetration have been underestimated previously and now are corrected in the SMARTer 2020. This however would make referencing the old report obsolete, as in fact it hasn’t been used to get the value of 0.19 GtCO$_2$e.

Another important point to consider is how much of an overlap there is between optimisation of truck route planning and optimisation of logistics network. While they are two separate sublevers and their effects are different, an increase in use of logistics network will drive an increase in route planning, and vice versa. When one of those technologies is implemented, the other is likely to follow, at least to some extent. Therefore, the savings potential for this sublever might be higher than estimated, due to positive impact from optimisation of logistics networks. Although the amount of impact will be difficult to determine, especially as it is indirect.


%\pagebreak