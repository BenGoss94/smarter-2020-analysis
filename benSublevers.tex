\subsection{Telecommuting}

%Introduction
Telecommuting is a method of completing ones job duties without being in the primary location of employment. This can be done through various means, and some of the main methods include using remote desktop applications to access the work files, folders and programs from another location; using email and instant messaging clients to ask questions of other employees rather than talking face to face; and also using phone calls and conference calls rather than scheduled meetings. 

%Data assumption
In the SMARTer 2020 report, the abatement potential of telecommuting is stated as 0.26 GtCO$_2$e. The only reference that is given for this is that the value is ``Scaled from global SMART 2020 reported based of ratio of India transportation emissions to global total in 2020 (4.8\%)". From tracking back through the SMART 2020 report to find these values, it is possible to see that the only information given is from Figure 9 on page 30. This figure shows the abatement potential of the different levels that relate to telecommuting, and the only reference given is that the values have come from "Expert Interviews, Jan – March 2008", which gives very little information about the numbers that are found. From searching through the text in the report, a list of assumptions that are made when they calculate this value is shown. The first assumption is that work-related car travel in urban and non-urban areas decreases by 80\%, and non-work-related car travel increases by 20\%. This assumption is vague; there is no reference to why the assumption is made, and the values that are chosen are not referenced within the text.
The next assumption made is that in developed countries, 10\% of existing vehicles are affected, equivalent to 20\% of people and 30-40\% of working population, and 7\% in developing countries. This assumption is not explained in enough detail. There are no explanation as to what it is affected by, and nowhere in the rest of the report are the values used. There is no reference anywhere in the report for these numbers.
The last stated assumption is that there is a 15\% increase in residential building emissions and a 60\% reduction in office emissions, applied to 10\% of residential buildings and 80\% of office buildings. This assumption is wrong as on page 31, they state that there could be an energy saving of between 20 and 50\%, which is not the value they used for their assumption. Even this value used is not referenced correctly, and the location of the original data is unknown.

In the SMARTer 2020 report, for uptake issues that come with telecommuting, some of the stated examples are that the companies may be concerned about the loss of productivity, or workplace culture deterioration. Workplace culture deterioration is quite a large issue, as if employees don't feel that they enjoy working, they're less likely to get the same amount of work completed as their office working equivalent. This also creates a loss of productivity. It is also stated that an issue is paying for the ICT structure necessary to handle such a scheme. This creates a financial burden for the company, whereas having an office and everyone travelling to it is more cost effective than creating the IT network. The report states that this could be improved by inducing employers with some kind of tax benefit, which would offset the cost of purchasing the ICT systems. 


\subsection{Applications for Intermodal Travel/Public Transportation}
\label{sec:intermodal}
Applications for intermodal transport are apps that could recommend the best way to get to a destination, or even tell a user when the next bus is arriving at a bus stop.
%Abatement Potential 0.07
The abatement potential of 0.07 GtCO$_2$e for this sublever is found using a value for the addressable emissions of 7.17 GtCO$_2$e, and a savings potential of 1\%. 
%Addressable emissions 7.17 IEA
The addressable emissions value is said to come from the IEA. Searching through their entire data catalogue is too extensive; they have various reports and databases that this value could come from. Thus, just referencing this value as IEA is unsuitable. 
%Savings potential 1% SMART 2020
The value for the savings potential is found in the previous report, SMART 2020, on page 67. It is used in a single figure, and then not mentioned again. In this report, it is labelled as an assumption rather than factual data, which creates uncertainty in the reliability of the data. This could be easily rectified if they referenced where the assumption came from correctly. 


\subsection{Soil Monitoring/ Weather Forecasting}
%Introduction
Soil monitoring is using ICT equipment to monitor levels of chemicals in the soil, to see what farmers need to add. It makes maximising yield a lot easier. Weather forecasting can be used in conjunction with this, to allow farmers to know how much pesticides or fertilisers to use. 
%Abatement potential 0.62
The abatement potential value for this sublever is found in the report to be 0.62 GtCO$_2$e. This is found using the addressable emissions value of 12.4 GtCO$_2$e, and a savings potential of 5\%. 
%Addressable emissions 12.4
%Addressable emissions defined as all agricultural emissions.1 
%1 EDGAR Database http://edgar.jrc.ec.europa.eu
In the information about the model, it is stated that the addressable emissions value used is the same for all agricultural emissions. This is an assumption that cannot be used lightly, as it leaves a large amount of uncertainty in the value. There are a large number of agricultural emissions that would probably not be affected by an increase in soil monitoring or weather forecasting, and using the assumption that this would affect the entirety of agriculture can be seen as presumptuous. 
The value provided is stated to have come from the EDGAR database, which is a database created by the Joint Research Centre of the European Commission. This database is extensive, so it is incredibly difficult to pinpoint the precise location of a single piece of data without an extensive search. 
An assumption that the SMARTer 2020 report makes is that the addressable emissions for global agriculture is the same as that of Europe, as the data chosen for the report only takes that in to account. It is not stated as to whether the data found from the EDGAR database is manipulated in any way to account for the change from continent to global. This assumption causes uncertainty within the results, and means that the data shown may not be a suitable value.

%Savings potential 5% Savings potential is based on case studies and expert interviews2 
%2 BSR "Wireless and the environment"
The value for the savings potential (5\%) is stated to have come from a report, known only as ``BSR Wireless and the Environment", and also case studies. These case studies are not referenced in the text, and without the location of this data it is incredibly difficult to justify the data. The section of the report that this value has come from is not specified, and from extensive search using key words the value of 5\% cannot be found. This leads to the assumption that the data has been created from somewhere in the report, and an expert has manipulated it in some way to give 5\%.

%GM Crops are cheaper
When implementing soil monitoring or weather forecasting methods in to agriculture, one of the main stated issues is that GM crops are generally cheaper, and are resistant to pesticides and other chemicals. This means they are also an easier alternative to tracking using sensors and other equipment. This is a fairly large issue, but the many members of the public have issues with GM crops, and feel strongly that they shouldn't be used, so knowing that the crops have come from different methods would probably be highly beneficial.
%Education - farmers don't know about the technology available for them
Another issue with implementing this kind of structure is that the farmers aren't aware of the technology that is available to them. This means that it isn't implemented as easily as in other areas. This is a major issue, and solving this could possibly involve an education system.
%Infrastructure - ICT network is common in developed countries, but less common in undeveloped and harder to implement
The report also states that the ICT networks in undeveloped countries are not as extensive as those in developed countries, meaning that this method is not viable in those countries. This is true, but if ICT networks and these kinds of methods are implemented as they become viable, rather than after other methods have already been used, this could cause the uptake from farmers to be much higher.


\subsection{Integration of Renewables in Commercial and Residential Buildings}

Integration of renewables in commercial and residential buildings is the use of renewable energy in the power generation for the buildings. It is becoming more and more common over time.

%Abatement potential 0.5

The abatement potential is stated to be 0.5 GtCO$_2$e, and this is made up of addressable emissions of 16.52 GtCO$_2$e, and from the savings potential of 3\%.
%addressable emissions 16.52 River Network, “Carbon Footprint of water” and IEA
The addressable emissions value is stated to have come from a report, called River Network, “Carbon Footprint of water”. This report deals in data to do mainly with water. Which means that the data provided more than likely comes from water heaters, washing machines and dishwashers, and not data to do with things that are used more often, like computers or televisions. The data is also stated to be partly created from data from the IEA, which, as stated in Section \ref{sec:intermodal} is incredibly difficult to trace. 


%savings potential 3% BSR “Wireless and the environment”

The value for the savings potential is 3\%, and is found in a report called ``Wireless and the Environment", by BSR (Business for Social Responsibility). Their aim is for a more sustainable world. The issue with this value and it's reference is that, throughout the report, it is incredibly difficult to find the information that states the savings potential is 3\%. The way to improve this is to reference in more depth, stating the chapter that this value actually came from.

%Financing - more expensive to implement
The fact that this method is more expensive to finance is referenced in the report. This is fairly accurate, but if a building is designed to have these features then it is cheaper than adding them afterwards. 

%Lack of familiarity - people dont know about it so dont think its viable
The lack of familiarity could easily be counteracted by supplying more information to the general public about the benefits of renewable energy for their buildings. 
%Landlord issues - many people rent houses, so adding this technology is something the landlord would have to implement, and quite often they are reluctant to
The fact that landlords would be reluctant to implement this technology is a large one, and is quite accurate. 





\subsection{Integration of Electric Vehicles}

Electric vehicles are becoming more and more popular as a concept, but the actual use of them in everyday life is not as common as vehicles that are fuelled by other means, such as hybrid, petrol or diesel. The electric vehicles have less of an environmental impact than vehicles using fossil fuels, as there are barely any emissions from the electric vehicles. The main source of emissions for electric vehicles are from the generation of the electricity, rather than the use of the vehicles. This is in contrast to generically-fuelled vehicles, which have the majority of their emission from usage. 

%Abatement potential - 0.2

The abatement potential calculated for the integration of electric vehicles is 0.2 GtCO$_2$e, and this is calculated using the addressable emissions of 9.56 GtCO$_2$e and savings potential of 2.1\%.
%Addressable emissions - 9.56
%Determined from all road transportation
%IEA

The addressable emissions are found from the IEA database. As can be seen from Section \ref{sec:intermodal}, the IEA database is so large that without an incredible amount of searching the data cannot be found.
%Savings potential - 2.1\% 
%The savings potential is calculated from average reduction in emissions of 30% in the US, number assumed to be fairly constant world-wide (although there are some differences due to different generation mix)2 and a 7% penetration3 

%New York Times: “How green are electric cars depends on where you plug in”
%www.nytimes.com/2012/04/15/automobiles/how-green-are-electric-cars-depends-on-where-you-plug-in.html?pagewanted=all&_r=0
%BCG Perspectives: “Batteries for Electric cars”
The main issue with the references that they have associated with the value of 2.1\% for the savings potential is that in the New York Times article “How green are electric cars depends on where you plug in”, the authors have picked out a value of 30\%. This value is not mentioned anywhere in the article. The article could have been rewritten more recently, or they could have taken the value from a different source and referenced the wrong one. An issue with the 30\% is that the authors specifically stated that the estimate is for emissions in the US, and it has then been assumed to be the constant rate worldwide. This is assumed regardless of the fact that it could be wrong, because the US is one of the higher emitting countries in the world. Thus, the value in question could instead be lower. 
After the 30\%, they multiply this by 7\%, to find the savings potential. This value is referenced as being from a report, from BCG Perspectives called "Batteries for Electric Cars". The actual value is not located in the report, after extensive search this value could not be found. This creates a large amount of uncertainty in the report, as the authors have obviously used a different piece of data and manipulated it in another way to get this value. 


%Culture around car ownership
Using the fact that there is a culture around car ownership in certain countries does not mean that people will be uninterested in using more emissions reducing transport. Generally, if people care about their cars, they will be more open to leaving it at home and taking care of it and using another mode of transport than putting it at risk every day by driving it.




% Uncertainty analysis consists of:
% 		Quality of data used
% 		Model Structure
% 		obstacles and enabling factors that might affect the uptake of innovations 					(financial, technical, social)
% 		Implicit assumptions in boundaries and methodology (for example rebound effect)